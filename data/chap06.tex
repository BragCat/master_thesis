% !TeX root = ../main.tex

\chapter{总结与展望}
\label{future}

  \section{工作总结}
  \label{future:summary}
  网络层源地址真实性验证的缺乏,是网络体系结构设计中的一大漏洞,成为了其安全问题的一大根源,被众多网络攻击手段加以利用。缺乏网络层源地址的真实性保障,一方面导致网络中恶意用户的攻击行为难以防范与阻止,另一方面为恶意用户的审计与追责带来了重重困难,使得攻击者可以无视低廉的犯罪成本肆无忌惮地发起网络攻击,难以对网络攻击形成有效威慑机制。SAVA的研究为网络层源地址伪造的问题带来了系统性的解决方案,实现主机粒度防地址伪造的SAVI技术,更是为用户身份识别与溯源技术研究提供了基础。但是,SAVI技术在实际网络中的部署推广还面临接入网络环境复杂、设备支持不足的困难,域间源地址验证方案中的众多设计也存在着共有的中心化风险。

  在IPv6源地址验证之上,NIDTGA地址生成方案提出将网络用户的身份信息加密后嵌入到网络层IPv6地址中,拥有密钥的审计者可以直接根据IPv6地址对网络中每个报文的发送者在现实世界中对应的实体信息进行追溯,而不必像普通接入网络一样在管理用户时维护地址与身份的绑定关系,避免了审计的管理复杂性、绑定关系维护与存储开销,拓展了对用户历史网络行为进行追责的时间尺度,从而对网络中的恶意攻击者形成更强力的威慑。

  但是,NIDTGA地址生成方案与其应用实现、部署之间有着巨大的鸿沟,其将IPv6地址与用户身份进行耦合,因而需要相应的认证手段进行配合。而实际网络环境中用户认证手段多种多样,802.1X、Web Portal、PPPoE等多种成熟的技术均有广泛应用,要进行用户身份识别与追溯系统的应用推广必须对各种认证手段下的实现进行具体的设计。同时,IPv6地址的配置方式又有着DHCPv6、SLAAC与静态配置之别,SLAAC的地址配置方式难以支持网络管理者对设备生成IPv6地址方法的操控。不同的操作系统、设备厂商对于各种认证方式、地址配置方式的支持又有所差异,为用户身份识别与溯源系统的设计与应用增添了新的困难。

  此外,当用户身份识别与溯源系统面临多个管理域的全局溯源问题时,其统一管理的各组织密钥历史的安全性将受到考验。一方面,各个组织的密钥历史需要进行集中式的管理与权限控制,另一方面,集中式追溯服务器的设计又将引入新的中心化问题。

  因此,本文从IPv6源地址验证技术的增强方案开始,对用户身份识别与溯源技术进行了自底向上的研究,主要贡献如下:
  \begin{enumerate}[1{)}]
    \item \textbf{提出了防止接入子网源地址伪造的SAVI增强部署方案与域间源地址验证SMA联盟注册的区块链设计方案RegChain。} \\
      对于用户身份识别与追溯系统所依赖的源地址验证技术,本文以其中较为重要的接入子网与自治域间源地址验证技术为研究对象,分别分析了在接入子网SAVI技术与自治域间SMA方案设计与部署时存在的问题,并提出了相应的安全性增强方案。对于SAVI技术,其局限性在于必须要求在第一跳接入交换机处部署才能够取得接入子网内完全防止地址伪造的效果,这一要求对在实际网络环境中进行推广部署SAVI技术过于严苛。针对难以满足这一理想情况的网络环境,本文提出了基于VLAN划分的SAVI增强部署方案,以彻底杜绝接入子网内发生源地址伪造。在自治域间源地址验证技术这一层次,本文分析了基于端到端标签合法性验证思路的一类方案所共有的安全缺陷,指出其引入的额外信任基础所带来的巨大安全隐患,并以其中典型的SMA技术为例,设计了其基于区块链的安全联盟注册中心方案,消除了多个自治域所组成的安全联盟系统中具有关键作用的单一故障点,以多副本备份数据、多节点提供服务、区块链保护数据真实的RegChain方案作为替代,避免了联盟注册中心功能失效而导致联盟内所有跨域报文转发均出现故障的风险,为所有端到端类域间源地址验证技术提供了一致的思路指引,统一解决了域间源地址验证技术中一类重要方案共有的安全问题。
    \item \textbf{设计并实现了各种网络环境中的用户身份识别与溯源系统,提出了在校园网内的部署方案。} \\
      为了NIDTGA地址生成方案付诸实践,实现在实际网络环境中能够得到部署与应用的用户身份识别与溯源系统,本文研究了IPv6地址DHCPv6、SLAAC与静态配置三种配置方式下基于各种认证手段的用户身份识别与溯源系统设计方案,完成了用户设备接入上网场景的全覆盖。在每种地址配置方式与认证手段的组合下,本文均对其设计方案进行了实现,多种方案在现网环境中获得了部署与使用。此外,在经过对多种方案进行分析与对比后,本文选择最适宜推广的二层准入认证的系统设计方案,将其在校园网络中部署时对网络环境的要求进行总结,架设了NIDTGA地址生成方案从理论到应用落地之间的桥梁,为用户身份识别与溯源系统在CNGI-CERNET2连接的41所高校校园网络中进行推广应用提供了指南。
    \item \textbf{提出了基于区块链的用户身份溯源系统设计方案NIDChain。} \\
      本文分析了用户身份识别与溯源系统在大规模部署后跨组织溯源时存在的安全性问题,指出用户身份追溯这一环节的薄弱,容易发生各组织密钥的泄漏、密钥历史的篡改、溯源功能的不可用等问题,威胁用户隐私,影响用户身份识别与溯源系统在全局范围内进行多组织的用户身份溯源功能正常发挥作用。利用区块链防数据篡改与分布式高可用的特点,本文提出了采用区块链构建用户身份溯源系统的NIDChain方案,在使用区块链存储各组织密钥历史密文的同时,确保仅有审计方能够获取组织密钥历史,防止组织密钥泄漏,同时提供了防止密钥历史遭到篡改的真实性保护。NIDChain网络的多节点也为用户身份溯源系统抵御拒绝服务攻击提供了保障,从数据安全、隐私保护、容灾备份、服务高可用等多个方面提升了用户身份识别与溯源系统的可用性。
  \end{enumerate}


  \section{未来展望}
  \label{future:perspective}
    未来可在下列几个方向开展本文后续的研究:
    \begin{enumerate}[1{)}]
      \item \textbf{NIDTGA地址生成方案对Android等操作系统设备的支持。} \\
        尽管本文中提出了NIDTGA地址生成方案在SLAAC方式下实现用户身份绑定与溯源的方案,并对Web Portal与二层认证两种方式均进行了相应的系统设计,但这仍局限于原有的接入用户管理模式,仅仅作为一种实现设备全覆盖的折中方案,没有充分发挥NIDTGA地址的优势。Android手机不支持DHCPv6协议的现状导致了其难以使用需要身份管理和溯源控制的NIDTGA地址生成方案。除了开发定制DHCPv6客户端外,如何更好地解决用户身份识别与溯源系统在SLAAC配置方式下向用户设备IPv6地址中嵌入网络身份的问题,以及如何推动广泛使用的操作系统支持DHCPv6配置地址,以更好地实现对所有设备的统一支持,是后续研究的一大重点。
      \item \textbf{用户身份识别与溯源系统在全国高校的推广部署。} \\
        目前,用户身份识别与溯源系统基于二层认证的实现方案已在清华大学无线校园网中进行了部署。作为一项网络基础技术,将其推广至CNGI-CERNET2接入的全国各大高校,乃至全国运营商等更大范围部署,以发挥其对恶意攻击者形成威慑的作用,是后续的重点工作。一项技术的部署会增加网络管理员的运维成本,如何提高用户身份识别与溯源系统部署后带来的收益,比如针对部署区域实施特殊的资源访问政策等,是推广部署工作中的研究点。
      \item \textbf{源地址验证技术及其安全性。} \\
        SAVA中提出的接入子网、自治域内与自治域间三个层次的源地址安全,均对用户身份识别与溯源系统的安全性有着至关重要的影响。目前无线SAVI技术目前仍停留在IETF草案阶段,尚未形成真正的国际标准。自治域内的源地址验证技术缺乏充分的研究,并且大量的自治域间源地址验证技术也由于其方案或缺乏部署激励、或不支持增量部署而尚未得到大规模的应用。源地址验证技术能够从体系结构的角度一劳永逸地解决互联网中一大类安全问题,避免修补式的安全问题解决方案,继续推动其标准化、厂商支持、现网部署以及进一步的安全性研究是未来工作的方向。
    \end{enumerate}










